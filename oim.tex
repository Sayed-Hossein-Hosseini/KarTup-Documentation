\chapter{مدل‌سازی تعامل شئ}\label{senario-table}
برای مدل‌‌سازی تعامل شئ، ۵ گام وجود دارد که به ترتیب باید انجام شوند:
\begin{enumerate}
	\item
	جمع‌آوری اطلاعات درباره‌ی فرایند‌های کسب‌وکار موجود 
	\item 
	تبیین سناریو‌هایی برای گام‌های غیربدیهی از مورد کاربرد‌های گسترده
	\item 
	ساخت جدول سناریو
	\item 
	استنتاج نمودار توالی از جداول سناریو
	\item 
	مرور مدل‌های تعامل شئ
\end{enumerate}
که در ادامه، با شناختی که از کسب‌وکار تا به اینجای کار پیدا کرده‌ایم، ۷ سناریو و جداول مربوطه در کنار آن نمودار توالی را نشان داده‌ایم.


\clearpage
\section{سناریو و مدل تعامل شئ برای گام 6 از \uc{14}}
\subsection{سناریو تعامل شئ برای \say{نشان‌دار کردن آگهی}}
\setcounter{MainStepCounter}{4}
\mainstep{کارفرما روی دکمه‌ی \say{پرداخت از طریق درگاه بانکی} کلیک می‌کند.}

\beginmainstep{صفحه‌ی پرداخت، اطلاعات را با یک آبجکت \json به کنترل‌گر آگهی می‌فرستد.}

\majorstep{کنترل‌گر کارفرما، اطلاعات را به درگاه بانکی ارسال و نتیجه را دریافت می‌کند.}
\indent\patchstep{اگر نتیجه‌ی تراکنش موفقیت‌آمیز بود:}
\indent\indent\betastep{کنترل‌گر کارفرما، اطلاعات فرم آگهی را به \gdm ارسال می‌کند}
\indent\indent\betastep{کنترل‌گر آگهی پیغام \say{آگهی با موفقیت ثبت شد.} \\را در یک آبجکت \json ذخیره می‌کند.}
\indent\indent\betastep{کنترل‌گر آگهی، اطلاعات را به صفحه‌ی تایید ارسال می‌کند }
\indent\patchstep{اگر نتیجه‌ی تراکنش ناموفق بود:}
\indent\indent\betastep{کنترل‌گر کارفرما پیغام \say{پرداخت ناموفق بود، آگهی ثبت نشد.} را\\ در یک آبجکت \json می‌نویسد.}
\indent\indent\betastep{کنترل‌گر آگهی، اطلاعات را به صفحه‌ی پرداخت (برای پرداخت مجدد) ارسال می‌کند.}

\subsection{جدول سناریو}
\begin{table}[H]
	\caption{جدول سناریو \arabic{table}}
	\begin{adjustbox}{width=\textwidth}
		\begin{tabular}{|c|c|c|c|c|}
			\hline														
			\# & فاعل & کنش فاعل & دیگرداده‌ها/اشیا & شئ‌ای که کنش روی آن انجام می‌شود \\
			\hline
			\hline
			\sstep &
			کارفرما &
			کلیک می‌کند &
			دکمه‌ی پرداخت از طریق درگاه بانکی &
			در صفحه‌ی پرداخت \\
			\hline
			\sstep &
			صفحه‌ی پرداخت &
			ارسال می‌کند &
			اطلاعات در یک آبجکت \json &
			به کنترل‌گر کارفرما \\
			\hline 
			\sstep &
			کنترل‌گر کارفرما &
			ارسال می‌کند &
			اطلاعات &
			به درگاه بانکی \\
			\hline
			\sstep &
			\multicolumn{4}{|r|}{اگر نتیجه موفقیت‌آمیز بود}\\
			\hline
			\sstep &
			کنترل‌گر کارفرما &
			ارسال می‌کند &
			اطلاعات فرم آگهی &
			به \gdm \\
			\hline
			\sstep &
			کنترل‌گر آگهی &
			ذخیره می‌کند &
			پیغام \say{آگهی با موفقیت ثبت شد.}&
			در آبجکت \json \\
			\hline
			\sstep &
			کنترل‌گر آگهی &
			ارسال می‌کند &
			اطلاعات &
			به صفحه‌ی تایید پرداخت \\	
			\hline
			
			\sstep &
			\multicolumn{4}{|r|}{اگر موفقیت‌آمیز نبود}\\
			\hline
			\sstep &
			کنترل‌گر کارفرما &
			ذخیره می‌کند &
			پیغام \say{پرداخت ناموفق بود،‌ آگهی ثبت نشد.}&
			در آبجکت \json \\
			\hline
			\sstep &
			کنترل‌گر آگهی &
			ارسال می‌کند &
			اطلاعات &
			به صفحه‌ی عدم تایید پرداخت \\		
			\hline
			
		\end{tabular}
	\end{adjustbox}
\end{table}
\setcounter{MainStepCounter}{0}
\setcounter{SenarioCounter}{0}
\subsection{نمودار توالی}

\clearpage
\section{سناریو و مدل تعامل شئ برای گام ۲ از \uc{23}}
\subsection{سناریو تعامل شئ برای \say{جست‌وجوی آگهی}}
\mainstep{کاربر کلید‌واژه‌ی مربوطه را در قسمت نوار جستجو وارد می‌کند}

\beginmainstep{صفحه‌ی جستجو‌ی آگهی‌ها، عبارت جستجو را در قالب یک آبجکت \json به همراه دیگر پارامتر‌های جستجو (مثل مرتب‌سازی و فیلتر‌ها) به کنترل‌گر آگهی می‌فرستد.}

\majorstep{کنترل‌گر آگهی، این آبجکت \json را به \gdm ارسال می‌کند.}

\majorstep{اگر نتیجه جستجو:}
\indent\patchstep{چیزی نبود، \gdm یک آبجکت \none را به کنترل‌گر آگهی ارسال می‌کند.}
\indent\patchstep{در غیر این صورت، تمامی آگهی‌‌های پیدا شده را به \json اصطلاحا\\ \serialize می‌کند و به کنترل‌گر آگهی می‌فرستد.}

\majorstep{کنترل‌گر آگهی، نتیحه را دریافت می‌کند.}

\majorstep{اگر \none بود:}
\indent\patchstep{پیغام \say{آگهی‌‌ای پیدا نشد} را در یک آبجکت \json ذخیره و  صفحه‌ی نتایج جستجو ارسال می‌کند.}
\indent\patchstep{در غیر این صورت، آبجکت‌های \json دریافتی را به صفحه‌ی نتایج جستجو می‌فرستد.}

\subsection{جدول سناریو}
\begin{table}[H]
	\caption{جدول سناریو \arabic{table}}
	\begin{adjustbox}{width=\textwidth}
		\begin{tabular}{|c|c|c|c|c|}
			\hline								
			\# & فاعل & کنش فاعل & دیگرداده‌ها/اشیا & شئ‌ای که کنش روی آن انجام می‌شود \\
			\hline
			\hline
			\sstep &
			صفحه‌ی جستجو‌ی آگهی &
			ارسال می‌کند &
			آبجکت \json &
			به کنترل‌گر آگهی \\
			\hline
			\sstep &
			کنترل‌گر آگهی&
			ارسال می‌کند &
			آبجکت \json &
			به \gdm\\
			\hline
			\sstep &
			\multicolumn{4}{|r|}{اگر چیزی پیدا نشده بود}\\
			\hline
			\sstep &
			\gdm&
			ارسال می‌کند &
			آبجکت \none&
			به کنترل‌گر آگهی \\
			\hline
			\sstep &
			\multicolumn{4}{|r|}{در غیر این صورت}\\
			\hline
			\sstep &
			\gdm&
			\serialize می‌کند&
			آبجکت‌های پیدا شده &
			به \json \\
			\hline
			\sstep &
			\gdm &
			ارسال ‌می‌کند&
			آبجکت \json &
			به کنترل‌‌گر آگهی \\
			\hline
			\sstep &
			کنترل‌‌گر آگهی&
			دریافت می‌کند &
			\begin{inparaitem}
				\item \none 
			\end{inparaitem}
			یا 
			\begin{inparaitem}
				\item \json
			\end{inparaitem}
			&
			\\
			\hline
			\sstep &
			\multicolumn{4}{|r|}{اگر \none بود}
			\\
			\hline
			\sstep &
			کنترل‌گر آگهی &
			ذخیره می‌کند &
			پیغام \say{آگهی‌ای پیدا نشد}&
			در آبجکت \json \\
			\hline
			\sstep &
			کنترل‌گر آگهی &
			ارسال می‌کند &
			آبجکت \json &
			صفحه‌ی نتایج جستجو\\
			\hline
		\end{tabular}
	\end{adjustbox}
\end{table}
\setcounter{MainStepCounter}{0}
\setcounter{SenarioCounter}{0}
\subsection{نمودار توالی}


\clearpage
\section{سناریو و مدل تعامل شئ برای گام 2 از \uc{25}}
\subsection{سناریو تعامل شئ برای \say{مشاهده‌ی رزومه‌ها}}\label{senario-counter}
\mainstep{کارفرما بر روی دکمه‌ی \say{رزومه} در پروفایل کارجوی مدنظر کلیک می‌کند.}

\beginmainstep{صفحه‌ی پروفایل کارجو، یک درخواست مبنی بر خواست رزومه و اطلاعات کارجو برا به صورت \json به کنترل‌گر کارجو می‌فرستد.}

\majorstep{کنترل‌گر کارجو، این اطلاعات را به \gdm\RTLfootnote{\lr{Great Database Manager} این کلاس مسئول مدیریت مدل‌های موجود در \lr{ORM} معماری پروژه‌ است.} می‌فرستد}
\indent\patchstep{اگر رزومه‌ای وجود داشت، آن را در یک آبجکت \json برای کنترل‌گر می‌فرستد}
\indent\patchstep{اگر رزومه‌ای نبود، یک آبجکت \none به کنترل‌گر کارجو، برمیگرداند.}\\
\majorstep{کنترل‌گر کارجو، آبجکت \json پاسخ را دریافت می‌کند.}
\indent\patchstep{اگر پاسخ بازگشتی، \none نبود آبجکت \json را برای ارسال به صفحه‌ی پروفایل کارجو آماده نگه‌ می‌دارد.}
\indent\patchstep{در غیر این صورت، پیغام \say{عدم وجود رزومه} را در یک آبجکت \json  ذخیره می‌کند.}

\majorstep{کنترل‌گر آبجکت \json را به صفحه‌ی پروفایل کارجو می‌فرستد.}

% --------------------------------------------------------
\setcounter{MainStepCounter}{0}

\subsection{جدول سناریو}
\begin{table}[H]
	\caption{جدول سناریو \arabic{table}}
	\begin{adjustbox}{width=\textwidth}
		\begin{tabular}{|c|c|c|p{0.4\textwidth}|c|}
			\hline
			\# & فاعل & کنش فاعل & دیگرداده‌ها/اشیا & شئ‌ای که کنش روی آن انجام می‌شود \\
			\hline
			\hline
			\sstep & 
			کارفرما &
			کلیک می‌کند &
			دکمه‌ی رزومه &
			از صفحه‌ی پروفایل کارجو \\
			\hline
			\sstep & 
			صفحه‌ی پروفایل کارجو &
			ارسال می‌کند &
			\begin{inparaitem}
				\item درخواست رزومه‌ی کارجو
				\item آبجکت \json اطلاعات کارجو
			\end{inparaitem} &
			به کنترل‌گر کارجو \\
			\hline
			\sstep & 
			کنترل‌گر کارجو &
			ارسال می‌کند &
			\json &
			به \gdm \\
			\hline
			\sstep & \multicolumn{4}{|r|}{اگر رزومه‌ای وجود داشت} \\
			\hline
			\sstep & 
			\gdm &
			ذخیره می‌کند &
			رزومه &
			در \json\\
			\hline
			\sstep & 
			\gdm &
			ارسال می‌کند &
			آبجکت \json &
			کنترل‌گر کارجو \\
			\hline
			\sstep & \multicolumn{4}{|r|}{اگر رزومه‌ای وجود نداشت}\\
			\hline
			\sstep & 
			\gdm&
			بر می‌گرداند &
			\none&
			به کنترل‌گر کارجو \\
			\hline
			\sstep & \multicolumn{4}{|r|}{اگر \none بود} \\
			\hline
			\sstep & 
			کنترل‌گر کارجو &
			ذخیره می‌کند &
			متن \say{عدم وجود رزومه}&
			آبجکت \json\\
			\hline
			\sstep & 
			کنترل‌گر کارجو &
			ارسال می‌کند &
			آبجکت \json &
			به صفحه‌ی پروفایل کارجو \\
			\hline
		\end{tabular}
	\end{adjustbox}
\end{table}
\setcounter{MainStepCounter}{0}
\setcounter{SenarioCounter}{0}
\subsection{نمودار توالی}

\clearpage
\section{سناریو و مدل تعامل شئ برای گام 2 از \uc{17}}
\subsection{سناریو تعامل شئ برای \say{نشان‌دار کردن آگهی}}
\mainstep{کارجو روی علامت ستاره در صفحه‌ی مربوط به آگهی مدنظر کلیک می‌کند.}

\beginmainstep{صفحه‌ی آگهی، اطلاعات مربوط به آگهی، کارجو و درخواستی مبنی بر نشان‌دار کردن این آگهی را با یک آبجکت \json به کنترل‌گر آگهی ارسال می‌کند.}

\majorstep{کنترل‌گر آگهی از \gdm آبجکت کارجو را درخواست می‌کند.}

\majorstep{کنترل‌گر آگهی بررسی می‌کند که آیا این آگهی جزو آگهی‌های نشان‌دار شده‌ی این آبجکت کارجو هست یا خیر.}
\indent\patchstep{اگر آگهی جزو آگهی‌های نشان‌دار بود:}
\indent\indent\betastep{آن را از آگهی‌های نشان‌دار کارجو حذف می‌کند.}
\indent\indent\betastep{کنترل‌گر آگهی پیغام \say{آگهی از آگهی‌های نشان‌دار حذف شد.} را\\ در یک آبجکت \json ذخیره می‌کند.}
\indent\patchstep{در غیر این صورت:}
\indent\indent\betastep{آن را به آگهی‌های نشان‌دار کارجو می‌افزاید.}
\indent\indent\betastep{کنترل‌گر آگهی، پیغام \say{آگهی به آگهی‌های نشان‌دار افزوده شد.} را\\ در یک آبجکت \json ذخیره می‌کند.}

\majorstep{کنترل‌گر آگهی، آبجکت کارجو را به \gdm می‌فرستد.}

\majorstep{\gdm آن را در پایگاه داده ذخیره می‌کند.}

\majorstep{کنترل‌گر، آبجکت \json را صفحه‌ی آگهی ارسال می‌کند.}

\subsection{جدول سناریو}
\begin{table}[H]
	\caption{جدول سناریو \arabic{table}}
	\begin{adjustbox}{width=\textwidth}
		\begin{tabular}{|c|c|c|p{6cm}|c|}
			\hline											
			\# & فاعل & کنش فاعل & دیگرداده‌ها/اشیا & شئ‌ای که کنش روی آن انجام می‌شود \\
			\hline
			\hline
			\sstep &
			صفحه‌ی آگهی &
			ارسال می‌کند &
			\begin{inparaitem}
				\item آگهی 
				\item کارجو
				\item درخواستی مبنی بر نشان‌دار کردن
			\end{inparaitem}
			به کنترل‌گر آگهی &
			\\
			\hline
			\sstep &
			کنترل‌گر آگهی &
			درخواست می‌کند &
			آبجکت کارجو &
			از \gdm \\
			\hline
			\sstep &
			\gdm &
			ارسال می‌کند &
			آبجکت کارجو &
			به کنترل‌گر آگهی \\
			\hline
			\sstep &
			کنترل‌گر آگهی &
			وجود را بررسی می‌کند &
			آگهی &
			در آبجکت کارجو \\
			\hline
			\sstep &
			\multicolumn{4}{|r|}{اگر در لیست آگهی‌های نشان‌دار بود:}\\
			\hline
			\sstep &
			کنترل‌گر آگهی &
			حذف می‌کند &
			آگهی‌&
			از لیست آگهی‌های نشان‌دار کارجو \\
			\hline
			\sstep &
			کنترل‌گر آگهی &
			ذخیره می‌کند &
			پیغام \say{آگهی‌ از آگهی‌های نشان‌دار حذف شد.}&
			در آبجکت \json \\
			\hline
			\sstep &
			\multicolumn{4}{|r|}{در غیر این صورت}\\
			\hline
			\sstep &
			کنترل‌گر آگهی &
			اضافه می‌کند &
			آگهی‌ &
			به لیست آگهی‌های نشان‌دار کارجو \\
			\hline
			\sstep &
			کنترل‌گر آگهی &
			ذخیره می‌کند &
			پیغام \say{آگهی‌ به آگهی‌های نشان‌دار افزوده شد.}&
			در آبجکت \json \\
			\hline
			\sstep &
			کنترل‌گر آگهی &
			ارسال می‌کند &
			آبجکت کارجو &
			به \gdm \\
			\hline
			\sstep &
			کنترل‌گر آگهی &
			ارسال می‌کند &
			آبجکت \json &
			صفحه‌ی آگهی\\
			\hline
		\end{tabular}
	\end{adjustbox}
\end{table}
\setcounter{MainStepCounter}{0}
\setcounter{SenarioCounter}{0}
\subsection{نمودار توالی}

\clearpage
\section{سناریو و مدل تعامل شئ برای گام 2 از \uc{18}}
\subsection{سناریو تعامل شئ برای \say{مشاهده‌ی وضعیت آگهی‌های درخواستی}}
\mainstep{کارجو روی دکمه‌ی \say{وضعیت آگهی‌های درخواستی} در قسمت نوارابزار پنل کاربری کارجو کلیک می‌کند.}

\beginmainstep{صفحه‌ی پنل کاربری، درخواستی مبنی بر آگهی‌های درخواستی کارجو را به کنترل‌گر کارجو ارسال می‌کند.}

\majorstep{کنترل‌گر کارجو، شئ مربوط به آگهی‌‌های درخواستی را از \gdm درخواست می‌کند.}

\majorstep{\gdm شئ مربوط به آگهی‌های درخواستی را از پایگاه داده‌ می‌خواند.}
\indent\patchstep{اگر شئ‌ای موجود نباشد:}
\indent\indent\betastep{\gdm یک آبجکت \none ایجاد می‌کند.}
\indent\indent\betastep{\gdm شئ ساخته شده را به کنترل‌گر کارجو می‌فرستد.}\\
\indent\patchstep{اگر شئ مربوط به آگهی‌های درخواستی موجود باشد:}
\indent\indent\betastep{\gdm آن را در یک آبجکت \json ذخیره و به کنترل‌گر کارجو می‌فرستد.}

\majorstep{کنترل‌گر کارجو یک آبجکت \json یا \none دریافت می‌کند}
\indent\patchstep{اگر آبجکت \none بود:}
\indent\indent\betastep{کنترل‌گر کارجو پیغام \say{آگهی‌ درخواستی‌ای موجود نمی‌باشد} را در یک آبجکت \json ذخیره می‌کند.}

\majorstep{کنترل‌گر آبجکت \json را به صفحه‌ی پنل کارجو ارسال ‌می‌کند.}

\subsection{جدول سناریو}
\begin{table}[H]
	\caption{جدول سناریو \arabic{table}}
	\begin{adjustbox}{width=\textwidth}
		\begin{tabular}{|c|c|c|c|c|}
			\hline					
			\# & فاعل & کنش فاعل & دیگرداده‌ها/اشیا & شئ‌ای که کنش روی آن انجام می‌شود \\
			\hline
			\hline
			\sstep &
			کارجو &
			کلیک می‌کند &
			دکمه‌ی \say{وضعیت آگهی‌های درخواستی}&
			در قسمت پنل کارجو \\
			\hline
			\sstep &
			صفحه‌ی پنل کاربری &
			ارسال می‌کند &
			درخواست مبنی بر آگهی‌های درخواستی کارجو &
			به کنترل‌گر کارجو \\
			\hline
			\sstep &
			کنترل‌گر کارجو &
			درخواست می‌کند &
			شئ مربوط به آگهی‌های درخواستی &
			از \gdm\\
			\hline
			\sstep &
			\gdm &
			می‌خواند &
			شئ مربوط به آگهی‌های درخواستی &
			پایگاه داده \\
			\hline
			\sstep &
			\multicolumn{4}{|r|}{اگر شئ‌ای موجود نباشد}\\
			\hline
			\sstep &
			\gdm &
			ایجاد می‌کند &
			&
			\none را\\
			\hline
			\sstep &
			\gdm &
			ارسال می‌کند &
			\none &
			به کنترل‌گر کارجو \\
			\hline
			\sstep &
			\multicolumn{4}{|r|}{اگر شئ‌ای وجود داشت} \\
			\hline
			\sstep &
			\gdm &
			ارسال می‌کند &
			آبجکت مربوط به آگهی‌های درخواستی &
			به کنترل‌گر کارجو\\
			\hline
			\sstep &
			کنترل‌گر کارجو &
			دریافت می‌کند &
			\none یا \json &
			از \gdm \\
			\hline
			\sstep &
			\multicolumn{4}{|r|}{اگر شئ \none بود}\\
			\hline
			\sstep &
			کنترل‌گر &
			ذخیره می‌کند &
			پیغام \say{آگهی‌ درخواستی‌ای موجود نمی‌باشد.}&
			در آبجکت \json \\
			\hline
			\sstep &
			کنترل‌گر کارجو&
			ارسال می‌کند &
			آبجکت \json&
			صفحه‌ی پنل کارجو \\
			\hline
		\end{tabular}
	\end{adjustbox}
\end{table}
\setcounter{MainStepCounter}{0}
\setcounter{SenarioCounter}{0}
\subsection{نمودار توالی}


\clearpage
\section{سناریو و مدل تعامل شئ برای گام ۴ از \uc{12}}
\subsection{سناریو تعامل شئ برای \say{ارسال رزومه}}
\setcounter{MainStepCounter}{2}
\mainstep{کارجو رزومه‌ی خود را، بارگذاری می‌کند و به روی دکمه‌ی \say{ارسال} کلیک می‌کند.}

\beginmainstep{صفحه‌ی ارسال رزومه، فایل آپلود شده را برای کنترل‌گر \say{ارسال رزومه} می‌فرستد.
}

\majorstep{کنترل‌گر ارسال رزومه، فایل آپلود شده را بررسی می‌کند.}
\indent\patchstep{اگر فایل ارسالی، \lr{PDF} بود:}
\indent\indent\betastep{فایل را برای \gdm ارسال می‌کند.}
\indent\indent\betastep{پیغام \say{رزومه ارسال شد} را در یک آبجکت \json ذخیره می‌کند.}
\indent\indent\betastep{اطلاعات را به صفحه‌ی آگهی می‌فرستد.}
\indent\patchstep{در غیر این صورت:}
\indent\indent\betastep{پیغام \say{فرمت فایل ارسالی درست نیست، لطفا مجدداً تلاش کنید.} را در یک آبجکت \json ذخیره می‌کند.}
\indent\indent\betastep{اطلاعات را به صفحه‌ی ارسال رزومه می‌فرستد.}


\subsection{جدول سناریو}
\begin{table}[H]
	\caption{جدول سناریو \arabic{table}}
	\begin{adjustbox}{width=\textwidth}
		\begin{tabular}{|c|c|c|p{6cm}|c|}
			\hline		
			\# & فاعل & کنش فاعل & دیگرداده‌ها/اشیا & شئ‌ای که کنش روی آن انجام می‌شود \\
			\hline
			\hline
			\sstep & 		
			سیستم &			
			نشان دادن &
			صفحه‌ی بارگذاری رزومه &
			\\
			\hline
			\sstep & 		
			کارجو &			
			آپلود می‌کند &			
			فایل رزومه &			
			صفحه‌ی بارگذاری رزومه \\
			\hline
			\sstep & 		
			صفحه‌ی بارگذاری رزومه &			
			ارسال می‌کند &			
			\begin{inparaitem}
				\item فایل رزومه
				\item اطلاعات کارفرما
			\end{inparaitem}
			&			
			صفحه‌ی بارگذاری رزومه\\
			\hline
			\sstep & 		
			کنترل‌گر ارسال رزومه &			
			بررسی می‌کند &			
			فایل رزومه‌ی آپلود شده &			
			\\
			\hline
			\sstep & 		
			\multicolumn{4}{|r|}{اگر فایل بارگذاری شده، \lr{PDF} بود:}\\
			\hline
			\sstep & 		
			کنترل‌گر بارگذاری&			
			ذخیره می‌کند &			
			آبجکت \json&			
			\begin{inparaitem}
				\item فایل رزومه
				\item اطلاعات کارفرما
			\end{inparaitem}
			\\
			\hline
			\sstep & 		
			کنترل‌گر &			
			ارسال می‌کند &			
			آبجکت \json &			
			به \gdm\\
			\hline
			\sstep & 		
			کنترل‌گر&			
			ذخیره می‌کند &			
			متن \say{رزومه ارسال شد}&			
			در آبجکت \json \\
			\hline
			
			\sstep & 		
    		کنترل‌گر ارسال رزومه &
			ارسال می‌کند &			
			اطلاعات &			
			صفحه‌ی آگهی \\
			\hline
			\sstep & 		
			\multicolumn{4}{|r|}{اگر \lr{PDF} نبود:}\\
			\hline
			\sstep & 		
			کنترل‌گر &			
			ذخیره می‌کند &			
			پیغام \say{فرمت فایل ارسالی درست نیست، لطفا مجدداً تلاش کنید.}&			
			آبجکت \json \\
			\hline
			\sstep & 		
			کنترل‌گر &			
			ارسال می‌کند &			
			آبجکت \json &			
			صفحه‌ی بارگذاری رزومه \\
			\hline
		\end{tabular}
	\end{adjustbox}
\end{table}
\setcounter{MainStepCounter}{0}
\setcounter{SenarioCounter}{0}
\subsection{نمودار توالی}


\clearpage
\section{سناریو و مدل تعامل شئ برای گام ۶ از \uc{1}}
\subsection{سناریو تعامل شئ برای \say{ثبت‌نام کاربر}}
\subsection{جدول سناریو}
\begin{table}[H]
	\caption{جدول سناریو \arabic{table}}
	\begin{adjustbox}{width=\textwidth}
		\begin{tabular}{|c|c|c|c|c|}
			\hline								
			\# & فاعل & کنش فاعل & دیگرداده‌ها/اشیا & شئ‌ای که کنش روی آن انجام می‌شود \\
			\hline
			\hline
		\end{tabular}
	\end{adjustbox}
\end{table}
\setcounter{MainStepCounter}{0}
\setcounter{SenarioCounter}{0}
\subsection{نمودار توالی}
