\documentclass{book}

\usepackage[utf8]{inputenc}
\usepackage{graphicx}
\usepackage{hyperref}
\hypersetup{
	colorlinks=true,
	linkcolor=blue
}

\usepackage{pgf}
\usepackage{pgfpages}

\pgfpagesdeclarelayout{boxed}
{
	\edef\pgfpageoptionborder{0pt}
}
{
	\pgfpagesphysicalpageoptions
	{%
		logical pages=1,%
	}
	\pgfpageslogicalpageoptions{1}
	{
		border code=\pgfsetlinewidth{0.8pt}\pgfstroke,%
		border shrink=\pgfpageoptionborder,%
		resized width=.95\pgfphysicalwidth,%
		resized height=.95\pgfphysicalheight,%
		center=\pgfpoint{.5\pgfphysicalwidth}{.5\pgfphysicalheight}%
	}%
}

\pgfpagesuselayout{boxed}

\usepackage{xepersian}
\settextfont{Yas}

\title{
	% \includegraphics[width=3cm, height=3cm]{20230227_204029(1).jpg} \\
	{\Huge کارتاپ}
}
\author{
	پدید‌آورندگان: \\
	سینا ربیعی \\
	سید حسین حسینی \\
	فاطمه علی‌ملکی \\
	علی قدسی مآب \\
	زهره سورانی \\
	حانیه شمس الکتابی \\
	مهدی حق‌وردی
}

\date{اسفند ۱۴۰۱}

\setcounter{secnumdepth}{3}
\setcounter{tocdepth}{3}


\begin{document}
	\renewcommand{\bibname}{مراجع}
	\maketitle
	\tableofcontents
		
	\chapter{سند نیازمندی‌ها}
		\section{مقدمه} 
			با توجه به افزایش روز افزون نرخ بیکاری در کشور ما کاریابی به صورت چشم‌گیر مورد توجه تمامی اقشار جامعه قرار گرفته است. بدین منظور ایجاد یک سامانه هدفمند برای کاهش این نرخ، سودمند است. سامانه نرم افزاری \textbf{کارتاپ}، با معرفی کارجویان به کارفرمایان و توانمندسازی افراد به منظور دریافت کار، این نیاز مهم را براورده می سازد.
			\subsection{هدف}
				یکی از بزرگ‌ترین نیازهای جامعه امروز، یافتن شغل مناسب برای افراد است. در گذشته‌ای نه چندان دور، کارجویان برای پیدا کردن شغل، باید به دفاتر کاریابی مراجعه می‌کردند؛ اما مدتی‌ست که دیگر هر کاری از طریق اینترنت و به صورت آنلاین صورت می‌گیرد. با توجه به رقابت زیاد و اینترنتی شدن تمام امور، بهترین راه برای رفع این نیاز، طراحی پلتفرم کاریابی‌‌ای است که فضایی برای کارفرمایان و کارجویان فراهم می آورد تا بتوانند به راحتی به هدف خود برسند.
				
				سامانه‌ی کاریابی به این صورت است که مشاغل را در دسته‌يندی‌های متفاوتی به کاربر نمایش می‌دهد و با استفاده از فیلترها، کارجویان میتوانند لیست مشاغل مد نظر خود را بیابند. همچنین برای سهولت کاربران امکان ساخت رزومه با قالب‌های حرفه‌ای و آماده را برای کارجویان فراهم می‌کند. کارفرما‌ها می‌توانند با پرداخت مبلغی، آگهی خود را روی سامانه قرار دهند تا به افراد جویای کار نمایش داده شود. همچنین کارفرماها می‌توانند مهارت‌های مورد نیاز برای موقعیت شغلی مورد نظر و همچنین، نوع کار از لحاظ پاره وقت، تمام وقت ، دورکاری و... را مشخص کنند.
				علاوه بر موارد فوق این کار باعث شده تا نرخ بیکاری در کشور کاهش پیدا کند و افراد در کوتاه ترین زمان بتوانند شغل مورد نظر خود را پیدا کنند.
				
			\subsection{قلمرو} 
				این محصول که به نام کارتاپ شناخته می‌شود، بستری است که در آن متقاضیان کار می‌توانند شغل متناسب با مهارت‌های خود را جست‌وجو کنند و موقعیت‌های کاری مختلف را مقایسه کنند.
				
				همچنین کارفرمایان می‌توانند با توجه به نیازمندی‌های شرکت خود، آگهی شغلی ایجاد کنند.
				
				در کنار این موارد، بخش مهارت افزایی نیز وجود دارد که افراد می‌توانند با کسب آموزش‌های مورد نظر و کسب گواهی معتبر، خود را برای موقعیت‌های شغلی مختلف آماده کنند.

			\subsection{تعاریف، سرنام‌ها و کوته نوشته‌ها}
			\subsection{مراجع}
				برای بررسی مرجع استفاده شده به 
				\cite{kung2013object}
				مراجعه کنید.
			\subsection{طرح کلی}
				روند کار در سند تدوین شده به این صورت است که در ابتدا اهداف و ویژگی های محصول شرح داده می شود و سپس به واسط‌های مختلف (من جمله واسط‌های سیستم، کاربر، سخت‌افزاری،نرم‌افزاری و...)، کارکردهای محصول ،مشخصات کاربران سیستم، قیود، مفروضات و وابستگی‌ها پرداخته و در نهایت به نیازمندی‌های آن خواهیم پرداخت.  
				
		\section{شرح کلی}
			کارتاپ یک سیستم نرم‌افزاری برای کاریابی هدفمند در سازمان‌ها و شرکت‌هاست.
			
			از طریق این سامانه، کارفرما نیاز‌های استخدامی خود را مطرح نموده و سپس بر اساس شغل و قابلیت‌های اعلام شده، بایستی بتواند به طور هوشمندانه کارجویان مناسب را به وی معرفی نماید. به نحوی می‌توان گفت این سیستم به منظور هوشمندسازی حداکثری روال‌های سنتی در این زمینه است.
			
			از جمله امکانات این سیستم می‌توان به امکان ثبت نام کرفرما، ثبت اطلاعات اطلاعات شرکتی، اعلام نیاز استخدامی، ثبت آگهی و همچنین برای کارجویان، ایجاد پروفایل و رزومه شخصی اشاره نمود.
			
			\subsection{چشم‌انداز محصول}
				بر اساس سیستم مذکور درخواست‌های مورد نیاز برای کاربران با توجه به خواسته ارسال می‌شود و آن‌ها می‌توانند با بررسی درخواست‌ها و فایل‌های پیوست نظرات خود را اقدام کرده و در صورت نیاز با بکدیگر ارتباط بگیرند.
				
				از جمله امکانات این سیستم دریافت رزومه، درخواست اخذ تست‌های بالینی برای کارفرمایان و همچنین شرکت در تست‌های شخصیت شناسی، ساخت رزومه شخصی، انتخاب علایق شغلی برای کارجویان اشاره کرد.
				
				\subsubsection{واسط‌های سیستم}
					واسط‌های سیستم این مسئله را بیان می‌کند که ارتباط سامانه‌ی ما با سیستم‌های خارجی، از طریق چه واسطه‌هایی صورت می گیرد و چگونه با هم در تبادل اطلاعات مختلف هستند. به عنوان مثال:
					\begin{enumerate}
						\item 
	دسترسی به پایگاه‌داده‌ی اداره‌ی ثبت احوال برای احراز هویت کارجو‌یان، مورد نیاز است.
						\item 
	دسترسی به پایگاه‌داده‌ی اداره‌ي ثبت شرکت‌ها برای احراز هویت شرکت‌ها، مورد نیاز است.
						\item 
	از آنجایی که این پلتفرم کاربران زیادی خواهد داشت، به سرور‌های قدرتمند و سریعی جهت پاسخ به درخواست‌ها و انجام عملیات‌های لارم، نیاز داریم.
						\item 
	جهت ارتباط و اطلاع رسانی‌های مهم به کاربران از طریق پیامک، نیاز به ارتباط با سازمان‌های مخابراتی یا شرکت‌هایی‌ست که این نوع خدمات را ارائه می دهند.
					\end{enumerate}
				\subsubsection{واسط‌های کاربر}
					جهت استفاده‌ی صحیح و کارآمد کاربران از سامانه، یک سری قابلیت‌های عمومی برای همگان و یک سری قابلیت‌های خاص در پنل کاربری کاربرانِ وارد شده در حساب کاربری، وجود دارد. در نتیجه نقش کاربران تعیین کننده‌ی سطح دسترسی آن‌ها می‌باشد. 
					سطح‌ دسترسی یا نقش کاربران در این سامانه، به دو دسته تقسیم می شود:
					\begin{enumerate}
						\item کارفرما
						\item کارجو
					\end{enumerate}
				\subsubsection{واسط‌های سخت‌افزاری}
					واضح است سیستم نرم‌افزاری کاریابی، نیازهای سخت‌افزاری به‌خصوصی ندارد؛ با این وجود لیستی از واسط های سخت‌افزاری مورد نیاز اولیه در ادامه آمده است:
					\begin{enumerate}
						\item 
						ابزارهای اولیه جهت پردازش و مدیریت دادهها و عملیات:
							\begin{itemize}
								\item 
								کارت شبکه
								\item 
								مودم (اتصال اینترنت)
								\item 
								سرور شبکه
								\item 
								سرور پردازش داده
							\end{itemize}			
						
						\item 
						ابزار لازم برای پیدا کردن مکان دقیق شرکت‌ها:
						\begin{itemize}
							\item 
							سرویس \lr{GPS}
						\end{itemize}
						
						\item 
						دستگاههای موردنیاز جهت ارتباط افراد با بستر اینترنت (هر سخت‌افرازی که توانایی اجرای نرم‌افزارهایی نظیر مرورگرها را داشته باشد) مانند:
						\begin{itemize}
							\item تلفن همراه
							\item کامپیوتر شخصی
							\item تبلت
							\item لپ‌تاپ
						\end{itemize}
						
					\end{enumerate}
				\subsubsection{واسط‌های نرم‌افزاری}\label{software}
					\begin{itemize}
						\item 
						مرورگر‌های مرسوم همچون 
						\lr{Google Chrome}،
						\lr{Mozilla Firefox} و
						\lr{Microsoft Edge} 
						که از آخرین نسخه‌های 
						\lr{HTML}،
						\lr{CSS}
						و
						\lr{JavaScript}
						پشتیبانی می‌کنند.
						
			            \item 
			            با توجه به حجم بالای داده‌ها، استفاده از سیستم‌های پایگاه‌ داده‌ی رابطه‌ای \LTRfootnote{Relational databases} و پایگاه‌داده‌های غیر رابطه‌ای \LTRfootnote{NOSQL databases} 
			            \item 
			            هر نرم‌افزاری که بتواند فایل با فرمت \lr{PDF} را نشان بدهد.
					\end{itemize}
				\subsubsection{واسط‌های ارتباطی}
					این سیستم به صورت تحت‌ وب است که کاربران با توجه به نیاز‌ها با سرور و پایگاه داده ثبت احوال و اداره ثبت شرکت‌ها ارتباط گرفته تا احراز هویت شوند و کار مورد نظر خود را انجام دهند.
					
				\subsubsection{واسط‌های حافظه}
					از آنجا که در سیستم، لازم است اطلاعات ضروری کاربران که بخش اعظم جامعه را تشکیل می‌دهند، ذخیره و آمارگیری‌های مورد نیاز از طریق این داده‌ها استخراج شود، پس منطقی است که حافظه‌ی جانبی قابل توجهی به سیستم اختصاص یابد. همچنین در 
					پروسه‌ی تخصیص حافظه، نیاز سیستم به پردازش سریع داده‌ها در مراحل جستجو میان مشاغل در نظر گرفته شده است. 
					پس به طور کلی:
					
					\begin{enumerate}
						\item 
						باتوجه به حجم پردازشی بالای این وب‌سایت جهت انجام امور مختلف، این سامانه نیازمند \lr{CPU}های قدرتمند و به‌روز و همچنین حافظه‌های عظیم و پرسرعت (همانند \lr{SSD}) است.
						
						\item 
						همچنین از \lr{RAM}های قدرتمندی برای تسریع درخواست ها استفاده می‌شود.
					\end{enumerate}
				\subsubsection{واسط‌های عملیات}
					\begin{enumerate}
						\item 
						اطلاعات بین سامانه و پایگاه داده، به صورت خودکار تبادل می شود و به صورت دستی چیزی تغییر نمی‌یابد (مگر در صورت ایجاد مشکلی خاص.)
						\item 
						برای این سامانه، نیاز به سرورهای قدرتمند و سریعی برای پردازش و ذخیره سازی داده‌ها نیاز است.
						\item 
						مراحل اعتبارسنجیِ صحت اطلاعات ورودی و فیلترهای جست‌و‌جو به صورت خودکار، توسط سامانه انجام می‌شود.
						\item 
						تمامی اطلاعات ویرایش شده یا بارگذاری شده، در همان لحظه 
						(به صورت \lr{real time} \RTLfootnote{به سیستم‌‌‌هایی گفته می‌شود که به صورت بی‌درنگ و بدون نیاز به بارگذاری (\lr{reload}) مجدد صفحه‌، اطلاعات بروزشده نمایش داده می‌شوند؛ پیام‌رسان‌ تلگرام از بهترین مثال‌های این سیستم‌هاست.})
						 در سرور‌های سامانه بروزرسانی یا بارگذاری می‌شوند.
						\item 
						در صورت استفاده‌ی بیش از حد مجاز تعداد کاربران جهت متعادل سازی سامانه، باید از طریق هدایت ترافیک به چندین سرور، دسترسی به یک دامنه را آسان‌تر و سریع‌تر کرد.
						\item 
						ارسال پیامک‌های انبوه به کاربران جهت اطلاع رسانی‌های مهم، به طور خودکار توسط سیستم‌های ارائه دهنده‌ی این نوع خدمات، انجام می‌شود.
						\item 
						سامانه باید به صورت خودکار رزومه‌های کارجویان را با درخواست‌های شغلی کارفرمایان مقایسه کند و در صورت مطابقت به طرفین پیشنهاد دهد.
						\item 
						سامانه باید مهارت‌های کارجویان را از رزومه‌های آن‌ها به طور خودکار استخراج کند.
						\item 
						احراز هویت شرکت‌ها به صورت خودکار انجام شود.
					\end{enumerate}
				\subsubsection{نیازمندی‌های سازگاری با محیط نصب}
					این سامانه روی تمامی دستگاه‌هایی که دارای مرورگر مورد نیاز در \ref{software} اشاره شده است، قابل اجرا می‌باشد و نیازی به نصب ندارد.
			\subsection{کارکرد محصول}
				این سیستم که به منظور سهولت در روند استخدام افراد در شرکت‌ها و یا پیدا کردن شغل توسط کارجویان طراحی شده‌ است، دارای قابلیت‌های متنوع برای هرکاربر می‌باشد:
				\begin{enumerate}
					\item کارجویان
						\begin{itemize}
							\item 
							کشف فرصت‌های شغلی
							\item 
							معرفی شرکت‌ها و فرصت‌های شغلی موجود در هرکدام
							\item 
							آگاهی از مشاغل جدید
							\item 
							استفاده از فیلتر های پیشرفته برای یافتن مهارت، نوع ساعت کاری
							\item 
							رزومه ساز آنلاین با قالب های پیشرفته و حرفه ای
							\item 
							 ارتباط آسان با کارفرمایان
							\item 
							افزایش  مهارت‌های فردی کارجویان برای پیدا کردن شغل بهتر
							\item 
							آموزش قوانین حقوقی به کارجویان برای جلوگیری هرچه بیشتر ازکلاهبرداری‌های اینترنتی و شغلی
						\end{itemize}
					
					\item کارفرمایان
						\begin{itemize}
							\item 
							جذب نیرو و درج آگهی استخدام
							\item 
							امکان تحلیل و بهینه‌سازی آگهی با استفاده از آمار دقیق.
							\item 
							مدیریت رزومه‌های دریافتی در پنل شرکت
							\item 
							مدیریت وضعیت درخواست متقاضی از داخل سیستم و اطلاع‌دهی به کارجو.
							\item 
							معرفی و تبلیغ برند
							\item 
							جستجو در رزومه‌های دریافتی
							\item 
							یادداشت گذاری بر روی رزومه‌ها
							\item 
							انتشار رایگان آگهی‌ کارآموزی
						\end{itemize}
				\end{enumerate}
			
			از دیگر قابلیت‌های سیستم به موارد زیر میتوان اشاره کرد:
			\begin{itemize}
				\item بخش مقالات و اخبار برای افزایش اطلاعات کاربران
				\item همگام با اصول بهینه سازی برای موتورهای جستجو
			\end{itemize}
		
			\subsection{مشخصات کاربر}
				کاربران کارتاپ به دو دسته ی کارفرمایان و کارجوبان تقسیم می شوند:
				\begin{enumerate}
					\item 
					کارجویان
					
					این دسته از کاربران شامل افرادی از جامعه هستند که در جست‌وجوی کاری مطابق با مهارت‌ها، استعدادها و یا مدرک تحصیلی آن‌ها با توجه به شرایطی همچون محل اقامت، میزان ساعات کاری و... می‌باشد. از این دسته افراد انتظار می‌رود که علاوه بر دسترسی به اینترنت، توانایی کار با مرورگر، ثبت نام، بارگذاری یا تشکیل رزومه، احراز هویت و همچنین آشنایی با زبان فارسی را داشته باشند.
					\item
					 کافرمایان
					
					این دسته از کاربران شامل افراد یا شرکت‌هایی هستند که در صدد پذیرش یا استخدام کارجو می‌باشند. آنها پس از بررسی و پذیرش رزومه‌ی کارجویان، مهارت‌ها و شرایط موردنظر خود را با مشخصات کارجو سنجیده و در صورت تطابق، کارجو را استخدام می‌کنند. این دسته از کاربران علاوه بر انتظاراتی که از کارجویان می‌رود ،ملزم به دارا بودن کد ثبت شده‌ی شرکت نیز می‌باشند . 
				\end{enumerate}
				
			\subsection{قیود}
				\begin{enumerate}
					\item 
					دسترسی به کارتاپ باید به صورت شبانه روزی برای کاربران فراهم باشد.
					\item 
					واسط‌های کاربری کارتاپ باید شراط آسان و قابل‌فهمی را برای کاربران فراهم سازد.
					\item 
					کارتاپ باید در کمتر از ۱۸ ماه به مشتری تحویل داده شود.
					\item 
					 هزینه تحلیل، طراحی و توسعه ی کارتاپ مطابق بودجه پروژه باید حداکثر \lr{50,000,000,000} ریال باشد.
				\end{enumerate}
			\subsection{قوانین کسب‌و‌کار}
				\begin{itemize}
					\item 
					رمز شخصی به هنگام احراز هویت و رمز موقت برای هر بار ورود، به شماره تلفن همراهی که کاربر هنگام ثبت نام وارد میکند فرستاده میشود
					\item 
					با توجه به اجباری بودن بیمه، کارفرمایان موظف هستند که شرایط بیمه کردن کارجویان را فراهم سازند.
					\item 
					استخدام کارجویان توسط کارفرمایان در چارچوب قوانین اداره کار صورت می‌پذیرد.
					\item
					هر کارفرما برای ثبت شرکت باید دارای کد تایید شده توسط سامانه ثبت شرکت‌ها باشد.
				\end{itemize}
			\subsection{مفروضات و وابستگی‌ها}
				در این قسمت هر یک از عوامل موثر بر الزامات مندرج در \lr{SRS} که می‌توانند بر آن تأثیر بگذارند، آورده شده است:
				
				\begin{enumerate}
					\item وابستگی‌ها
					\begin{itemize}
						\item
						 به دلیل حجم بالای اطلاعات، سیستم به پایگاه داده‌های کلان داده وابسته است.
						\item
						 اطلاعات پایگاه داده‌های اداره ثبت شرکت‌ها در جریان‌های کاری سیستم، مورد نیاز است.
						\item 
						جهت ارتباط و اطلاع رسانی‌های مهم به کاربران از طریق پیامک نیاز به ارتباط با سازمان‌های مخابراتی یا شرکت‌هایی است که این نوع خدمات را ارائه می دهند.
					\end{itemize}
				
					\item مفروضات
					\begin{itemize}
						\item کاربر توانایی دسترسی به اینترنت و تسلط کار با آن را داشته باشد.
						\item کاربر از دستگاهی با قابلیت اتصال به اینترنت و اجرای مرورگر جهت استفاده از خدمات سامانه، برخوردار است.
						\item کاربر حداقل دانش مورد نیاز برای کار با دستگاه‌های هوشمند را دارد.
						\item مرورگر کاربر از جاوا اسکریپت پشتیبانی کند.
					\end{itemize}
				\end{enumerate}
			
		\section{نیازمندی‌های خاص}
			\subsection{نیازمندی‌های واسط خارجی}
			\subsection{نیازمندی‌های کارکردی}
			برای فهم راحت‌تر و چیدمان بهتر، نیازمندی‌ها به سه دسته‌ی پلتفرم، کارچو و کارفرما تقسیم شده‌اند.
			\RTLfootnote{
			این تقسیم‌بندی قرار نیست خیلی دقیق باشد، چون مفهوم مطالب در بعضی موارد خیلی بهم نزدیک هستند؛ این کار صرفا برای جداسازی موارد مشابه بهم صورت گرفته است.
		}
			
				\begin{itemize}
					\item کارجو
						\begin{enumerate}
							\item 
							کارتاپ باید امکان نشاندار کردن و ذخیره کردن آگهی‌ها را برای کارجویان فراهم سازد.
							
							\item 
							کارتاپ باید آگهی‌های پیشنهادی مطابق با اطلاعات کارجو را نمایش دهد.
							
							\item 
							کارتاپ باید قسمتی را به عنوان صفحه شخصی کارجو شامل پروفایل، اطلاعات شخصی، علایق و دسته‌بندی مشاغل داشته باشد.
							
							\item 
							کارتاپ باید امکان تغییر مشخصات شناسنامه‌ای، اطلاعات تماس و محل اقامت را داشته باشد.
							\item 
							کارتاپ باید قسمتی را به عنوان پنل کاربری برای نمایش آخرین وضعیت و روند تمامی درخواست‌ها، شامل:
							\begin{itemize}
								\item ارسال شده
								\item در حال بررسی
								\item دیده شده توسط کارفرما
								\item تایید یا رد درخواست
								\item علل تایید یا رد خواست
							\end{itemize}
							
							و همچنین روند تمامی پیشنهادهای دیگر کارفرمایان برای استخدام کارجو را ارائه کند.
							
							\item 
							کارتاپ باید توانایی ایجاد و تشکیل رزومه‌ی الکترونیکی (رزومه ساز) برای کارجویان را فراهم نماید.
							
							\item 
							کارتاپ باید قابلیت بارگذاری فایل رزومه را برای کارجویان فراهم نماید.
							\item 
							کارتاپ باید امکان فیلتر آگهی‌ها را بر حسب زمان انتشار آن‌ها فراهم آورد.
							\item 
							کارتاپ باید آگهی‌های فوری و آگهی‌های پیشنهادی را برای کارجو نمایش دهد.
							
							\item 
							کارتاپ باید امکان فیلتر کردن مواردی از قبیل نام استان و شهر، نوع مهارت‌ها و انتخاب نوع موقعیت شغلی را برای کارجویان فراهم سازد.
							
							\item 
							کارتاپ باید امکان فرستادن رزومه را به چندین آگهی به صورت همزمان را داشته باشد.
							\item 
							کارتاپ باید به هنگام ثبت درخواست کارجو، امکان وارد کردن حقوق پیشنهادی وی را فراهم کند. 
						\end{enumerate}
					
					\item کارفرما
						\begin{enumerate}
							\item 
							کارتاپ باید امکان بارگذاری عکس‌هایی از محیط کاری،فضای شرکت و... را برای کارفرمایان فراهم کند.
							
							\item 
							کارتاپ باید امکان بارگذاری موقعیت مکانی شرکت توسط کارفرما را فراهم سازد.
							\item 
							کارتاپ باید بتواند کارجوبان مناسب و مطابق با شرایط آگهی‌های شرکت‌ها را یافته و آنان با به کارفرما‌ها پیشنهاد دهد.
							
							\item 
							کارتاپ باید امکان وارد کردن اطلاعاتی نظیر شرایط کاری، دستمزد، جنسیت و انتظارات عمومی و تخصصی را توسط کارفرما را فراهم کند.
							
							\item 
							کارتاپ باید یک صفحه مربوط به اطلاعات شرکت، پرسنل شرکت، آگهی‌های فعال، آگهی‌های منقضی شده، تصاویر، درخواست‌های کارجویان و پیشنهاد‌های ارائه شده به کارجویان برتر را به طور کامل نمایش دهد.
							
							\item 
							کارتاپ باید امکان ایجاد اکانت پرمیوم و خرید اشتراک برای کارفرمایان جهت ثبت بیش از 10 آگهی و همچنین ایجاد دیگر امکانات را فراهم کند.
						\end{enumerate}
					
					\item پلتفرم
						\begin{enumerate}
							\item 
							کارتاپ باید برای ثبت نام کارجویان، اطلاعاتی را از قبیل نام و نام‌خانوادگی، تلفن همراه و ایمیل را از کاربر دریافت نماید.
							
							\item 
							کارتاپ باید هنگام ثبت درخواست کارجو، عملیات احراز هویت کارجو (شامل وارد کردن تاریخ تولد و شماره شناسنامه، نوع دسته بندی مشاغل، نوع قرارداد و استان) را فراهم کند.
							
							\item
							کارتاپ باید امکان بازیابی رمز عبور کاربر را در صورت فراموشی، از طریق شماره همراه و یا ایمیل ثبت شده در سامانه فراهم کند.
							
							\item 
							کارتاپ باید برای هر رمز موقت، اعتبار ۱ دقیقه ای قائل شود و بعد از این زمان رمز منقضی شود.
							
							\item 
							کارتاپ باید برای ایجاد آگهی اسختدامی توسط کارفرما، عملیات احراز هویت، شامل:
							\begin{itemize}
								\item نام شرکت
								\item شماره‌ی ثبت شرکت یا شماره ملی شرکت
							\end{itemize}
							را داشته باشد.
							
							\item 
							کارتاپ باید توانایی اسکن و بررسی پروانه کسب کارفرما را داشته باشد.
							
							\item 
							کارتاپ باید توانایی تشخیص تطبیق یا عدم تطبیق شماره‌ی پروانه‌ی کسب و کد ثبت شرکت یا داشته باشد.
							
							\item 
							کارتاپ باید برای ایجاد آگهی استخدامی توسط کارفرما، عملیات احراز هویت (شامل نام شرکت و شماره‌ی ثبت شرکت یا شماره ملی شرکت) را انجام دهد.
							
							\item 
							کارتاپ باید امکان خارج شدن از سامانه را برای کاربر فراهم کند.
							\item 
							کارتاپ باید توانایی پاک کردن اکانت را برای کاربران فراهم کند.
						\end{enumerate}
				\end{itemize}
			\subsection{نیازمندی‌های کارایی}
				\begin{enumerate}
					\item 
					سامانه باید توانایی پاسخ گویی هم زمان ۱۰۰۰۰ کاربر را داشته باشد.
					\item 
					سامانه باید برای ورود کاربران از کد \lr{CAPCHA} \RTLfootnote{\lr{CAPCHA} یا همان کپچا، نرم‌افزاری آنلاین برای تولید سوالات و آزمون‌هایی‌ست که انسان براحتی قادر به پاسخ‌گویی به آنهاست ولی کامپیوتر‌ها در حال حاضر، قادر به تشخیص و پاسخ به آنها نیستند. عبارت \lr{CAPCHA} مخفف عبارت \lr{Completely Automated Public Turing Test To Tell Computers and Humans Apart} است.} استفاده کند تا از اینکه فرد وارد شده ربات نباشد، اطمینان حاصل کند.
					\item 
					سامانه باید برای ثبت نام کاربران با استفاده از کد احراز هویت، هوبت افراد را تایید نماید.
					\item 
					سیستم پیامکی سامانه باید بتواند پیامک‌ها را حداکثر ظرف ۲۰ ثانیه برای کاربران ارسال کند.
					\item 
					سامانه باید طراحی کاربرپسند داشته باشد.
					\item 
					سامانه باید قابلیت چت آنلاین را با کارشناس مربوطه برای کاربر فراهم نماید.
					\item 
					کارتاپ باید در هرگونه مواجه شدن با خطا، چه از سمت کاربر و چه از سمت سرور، خطار را با جزئیات گزارش دهد، تا نیروهای فنی این مورد را در اولین زمان ممکن بازبینی و رفع کنند.
				\end{enumerate}
			\subsection{قیود طراحی}
			\begin{enumerate}
				\item 
				امکان بارگیری رزومه‌ها به فرمت \lr{PDF} برای کاربران فراهم باشد.
				\item 
				سامانه باید بر روی تمامی مرورگر‌های مرسوم همچون 
				\lr{Google Chrome}،
				\lr{Mozilla Firefox} و
				\lr{Microsoft Edge}  قابل اجرا باشد.
			\end{enumerate}
			\subsection{صفت‌های سیستم‌ نرم‌افزاری}
				\begin{enumerate}
					\item امنیت
						\begin{itemize}
							\item 
							استفاده از قابلیت‌های پنل کاربری، فقط باید توسط کاربران احراز هویت شده، قابل دسترسی باشد.
							\item 
							سامانه باید حافط اطلاعات شخصی کاربران باشد.
							\item 
							سامانه باید قابلیت پشتیبان‌گیری از اطلاعات سایت، که شامل اطلاعات کابران هم می‌شود، و همچنین توانایی بازیابی اطلاعات را داشته باشد.
							\item 
							به جهت افزایش و پایداری امنیت ارتباط سرور با سیستم کاربر، از پروتکل‌های امنیتی مانند \lr{SSL} و \lr{HTTPS} استفاده می‌شود.
							\item 
							سامانه باید در صورت دریافت درخواست‌های بیش از حد مجاز اقدام به مسدود سازی کاربر به طور موقت کند.
							\item 
							سامانه باید به طور لحظه‌ای اقدام به ذخیره‌ی اطلاعات تغییر یافته کند.
							\item 
							سامانه باید در شرایط خاص خطاها را متوقف کند.
						\end{itemize}
					
					\item در دسترس بودن
						\begin{itemize}
							\item 
							سامانه باید به طور شبانه روز به جز بازه‌ی اصلاحات دوره‌ای، قابل دسترسی باشد.
							\item 
							سامانه باید از طریق تمامی مرورگر‌های مرسوم مانند
							\lr{Google Chrome}، 
							\lr{Mozilla Firefox}،
							و
							\lr{Microsoft Edge}
							که از آخرین نسخه‌های 
							\lr{HTML}،
							\lr{CSS}
							و
							\lr{JavaScript}
							پشتیبانی می‌کنند، در دسترس باشند.
							\item
							قابلیت مشاهده‌ی آگهی‌های استخدامی، حتی در صورت عدم ورود به حساب کاربری وجود داشته باشد.
						\end{itemize}
					
					\item پشتیبانی
						\begin{itemize}
							\item 
							سامانه باید تیمی متشکل از پشتیبانان در زمینههای مختلف داشته باشد (به عنوان مثال پشتیبان فنی و پشتیبان روابط عمومی.)
						\end{itemize}
					
					\item رابط کاربری مناسب
						\begin{itemize}
							\item 
							سامانه باید دارای رابط کاربری مناسب باشد. به طوری که هم دارای زیبایی های بصری باشد (\lr{UI}) و هم استفاده ی کاربر از آن ساده و معلوم باشد (\lr{UX}.)
						\end{itemize}					
				\end{enumerate}
			\subsection{برنامه تکرار و برنامه‌ی مرحله}
			

\bibliographystyle{acm-fa}
\bibliography{Ref}
\end{document}
