\documentclass{book}

\usepackage[utf8]{inputenc}
\usepackage{graphicx}
\usepackage{hyperref}
\hypersetup{
	colorlinks=true,
	linkcolor=blue
}

\usepackage{xepersian}
\settextfont{Yas}

\title{
	% \includegraphics[width=3cm, height=3cm]{20230227_204029(1).jpg} \\
	{\Huge کارتاپ}
}
\author{
	پدید‌آورندگان: \\
	سینا ربیعی \\
	سید حسین حسینی \\
	فاطمه علی‌ملکی \\
	علی قدسی مآب \\
	زهره سورانی \\
	حانیه شمس الکتابی \\
	مهدی حق‌وردی
}

\date{اسفند ۱۴۰۱}

\setcounter{secnumdepth}{3}
\setcounter{tocdepth}{3}

\begin{document}
	\maketitle
	\tableofcontents
	
	\chapter{سند نیازمندی‌ها}
		\section{مقدمه} 
			با توجه به افزایش روز افزون نرخ بیکاری در کشور ما کاریابی به صورت چشم‌گیر مورد توجه تمامی اقشار جامعه قرار گرفته است. بدین منظور ایجاد یک سامانه هدفمند برای کاهش این نرخ، سودمند است. سامانه نرم افزاری \textbf{کارتاپ}، با معرفی کارجویان به کارفرمایان و توانمندسازی افراد به منظور دریافت کار، این نیاز مهم را براورده می سازد.
			\subsection{هدف}
			\subsection{قلمرو} 
			\subsection{تعاریف، سرنام‌ها و کوته نوشته‌ها}
			\subsection{مراجع}
			\subsection{طرح کلی}
		\section{شرح کلی}
			\subsection{چشم‌انداز محصول}
				\subsubsection{واسط‌های سیستم}
					واسط‌های سیستم این مسئله را بیان می‌کند که ارتباط سامانه‌ی ما با سیستم‌های خارجی، از طریق چه واسطه‌هایی صورت می گیرد و چگونه با هم در تبادل اطلاعات مختلف هستند. به عنوان مثال:
					\begin{enumerate}
						\item 
	دسترسی به پایگاه‌داده‌ی اداره‌ی ثبت احوال برای احراز هویت کارجو‌یان، مورد نیاز است.
						\item 
	دسترسی به پایگاه‌داده‌ی اداره‌ي ثبت شرکت‌ها برای احراز هویت شرکت‌ها، مورد نیاز است.
						\item 
	از آنجایی که این پلتفرم کاربران زیادی خواهد داشت، به سرور‌های قدرتمند و سریعی جهت پاسخ به درخواست‌ها و انجام عملیات‌های لارم، نیاز داریم.
						\item 
	جهت ارتباط و اطلاع رسانی‌های مهم به کاربران از طریق پیامک، نیاز به ارتباط با سازمان‌های مخابراتی یا شرکت‌هایی‌ست که این نوع خدمات را ارائه می دهند.
					\end{enumerate}
				\subsubsection{واسط‌های کاربر}
					جهت استفاده‌ی صحیح و کارآمد کاربران از سامانه، یک سری قابلیت‌های عمومی برای همگان و یک سری قابلیت‌های خاص در پنل کاربری کاربرانِ وارد شده در حساب کاربری، وجود دارد. در نتیجه نقش کاربران تعیین کننده‌ی سطح دسترسی آن‌ها می‌باشد. 
					سطح‌ دسترسی یا نقش کاربران در این سامانه، به دو دسته تقسیم می شود:
					\begin{enumerate}
						\item کارفرما
						\item کارجو
					\end{enumerate}
				\subsubsection{واسط‌های سخت‌افزاری}
				\subsubsection{واسط‌های نرم‌افزاری} 
				\subsubsection{واسط‌های ارتباطی}
				\subsubsection{واسط‌های حافظه}
				\subsubsection{واسط‌های عملیات}
				\subsubsection{نیازمندی‌های سازگاری با محیط نصب}
			\subsection{کارکرد محصول}
			\subsection{مشخصات کاربر}
			\subsection{قیود}
			\subsection{قوانین کسب‌و‌کار}
			\subsection{مفروضات و وابستگی‌ها}
		\section{نیازمندی‌های خاص}
\end{document}
