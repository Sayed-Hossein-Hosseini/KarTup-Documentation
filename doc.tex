\documentclass{report}

\usepackage[utf8]{inputenc}
\usepackage{graphicx}
\usepackage{hyperref}
\hypersetup{
	colorlinks=true,
	linkcolor=blue
}

\usepackage{xepersian}
\settextfont{Yas}

\title{
	% \includegraphics[width=3cm, height=3cm]{20230227_204029(1).jpg} \\
	{\Huge کارتاپ}
}
\author{
	پدید‌آورندگان: \\
	سینا ربیعی \\
	سید حسین حسینی \\
	فاطمه علی‌ملکی \\
	علی قدسی مآب \\
	زهره سورانی \\
	حانیه شمس الکتابی \\
	مهدی حق‌وردی
}

\date{اسفند ۱۴۰۱}

\setcounter{secnumdepth}{3}
\setcounter{tocdepth}{3}

\begin{document}
	\maketitle
	\tableofcontents
	
	\chapter{سند نیازمندی‌ها}
		\section{مقدمه} 
			با توجه به افزایش روز افزون نرخ بیکاری در کشور ما کاریابی به صورت چشم‌گیر مورد توجه تمامی اقشار جامعه قرار گرفته است. بدین منظور ایجاد یک سامانه هدفمند برای کاهش این نرخ، سودمند است. سامانه نرم افزاری کارتاپ، با معرفی کارجویان به کارفرمایان و توانمندسازی فراد به منظور دریافت کار این نیاز مهم را براورده می سازد.
			\subsection{هدف}
			\subsection{قلمرو} 
			\subsection{تعاریف، سرنام‌ها و کوته نوشته‌ها}
			\subsection{مراجع}
			\subsection{طرح کلی}
		\section{شرح کلی}
			\subsection{چشم‌انداز محصول}
				\subsubsection{واسط‌های سیستم}
				\subsubsection{واسط‌های کاربر}
				\subsubsection{واسط‌های سخت‌افزاری}
				\subsubsection{واسط‌های نرم‌افزاری} 
				\subsubsection{واسط‌های ارتباطی}
				\subsubsection{واسط‌های حافظه}
				\subsubsection{واسط‌های عملیات}
				\subsubsection{نیازمندی‌های سازگاری با محیط نصب}
			\subsection{کارکرد محصول}
			\subsection{مشخصات کاربر}
			\subsection{قیود}
			\subsection{قوانین کسب‌و‌کار}
			\subsection{مفروضات و وابستگی‌ها}
		\section{نیازمندی‌های خاص}
\end{document}
